\chapter{Introduction}
\label{nopart-intro}

\lettrine{D}{ans} l'inconscient collectif français, le classique est le livre ennuyeux qu'on étudie en classe et, même, qui traumatise les élèves\cite[\S 21]{guittet2021enseignement}. Un jour, un \enquote{franconami} avait emporté avec lui son exemplaire annoté de Candide, et il s'est révélé divertissant. C'est comme ça que ce cercle a commencé.

Pour la suite, \enquote{je} désignera l'animateur du cercle. Toute personne participant à une rencontre devient \emph{\ClubStatusName{aspirant}}. Par défaut, cette opportunité est offerte à ceux qui peuvent soutenir une conversation en français et à un total de 4 par rencontre, \ClubStatusName{admin} inclus.
% ---sinon, ils sont invités à seulement écouter---

Dans les archives de l'Institut National de l'Audiovisuel repose un court-documentaire intitulé \enquote{les bergers érudits\cite[\TimeFR{10}{33}]{ina1967sagesse}}. Y apportant peut-être l'explication au sévère constat du début, l'un d'eux énonçait cette règle: \enquote{c'est l'assimilation qui compte}. Faisons-en la devise de ce cercle---excusant ainsi ma lecture d'escargot.

% Lire, méditer surtout. La lecture en elle même, c'est \dots pas grand chose. C'est un point de départ, mais surtout l'assimilation qu'on en fait qui compte. Ce n'est pas la qualité, ni même peut-être la qualité, mais l'usage qu'on en fait. Le profit, la maturation, et puis ce qu'il en sort après.

Encore faut-il utilement canaliser cet élan propre, loin des concours académiques.  Il ne s'agirait pas de s'improviser en Jack Kerouac du vagabondage intellectuel : l'on s'attachera en priorité à suivre des aides à l'étude\endnote{Encore faut-il mettre la main sur les aides à l'étude; tâche compliquée par la suppression de \foreignquote{english}{Inter Library Loan}\cite{larson2025alachua}\dots}; elles offrent le meilleur rendement. %Quitte à vagabonder, que ce soit dans les chemins de traverse uniquement, et sans lâcher son fil d'Ariane.

Pour satisfaire la notion d'assimilation, une idée ne vaut que par les liens que l'on tisse avec d'autres idées. Il y a, en français, une expression toute indiquée pour exprimer cela: \enquote{avoir de la suite dans les idées}. %Dit encore autrement: ruminer, mais sans s'enliser.
Dussé-je être comparé au moniteur de ski des Bronzés (avec son \enquote{planté de bâton}\cite{wiki:Les_Bronzés_font_du_ski}), j'aurai bien l'occasion de rappeler ce point!

Condition nécessaire à l'assimilation : l'assiduité. À cet égard, pour être \emph{\ClubStatusName{confirmed}}, il faut avoir été présent à 5 rencontres sur 12 mois glissants.

Comment s'assurer que la devise est suivie ? Je ne connaîs qu'une méthode : par introspection sur le fait que nos lectures ont enrichi, ou non, notre façon de penser. C'est pourquoi, à chaque rencontre, je demanderai \enquote{une phrase qui vous a marqué} des épisodes précédants.

Que lire? Se reporter au \RefSection{corpus-publ}. Là, la fiche de chaque œuvre indique un statut de lecture. Concrètement, au \DTMdate{2025-11-29}, il y a de quoi tenir un moment avec \hyperref[corpus-publ:lerouge]{Le rouge et le noir}.
%  Que lire? Se reporter au \RefChapterFR{corpus-publ}. Là, la fiche de chaque œuvre indique un statut de lecture. Concrètement, au \DTMdate{2025-11-29}, il y a de quoi tenir un moment avec \hyperref[corpus-publ:lerouge]{Le rouge et le noir}.

Ce projet, c'est aussi un patrimoine numérique en source ouverte \href{https://github.com/erwannr/cerclefl}{hébergé sur GitHub}, rédigé en \LaTeX---la référence en matière d'édition académique---et tendant vers la réutilisabilité (modulaire, etc.).

Comment contribuer ? L'on voudrait vivre d'amour et d'eau fraîche. Hélas, l'entretien de langue française présente un coût d'opportunité. Faire œuvre de bienfaisance→\URLgofundme, c'est ici la propager; vous devenez \emph{\ClubStatusName{donor}} (si, si\cite{larousse-passeur}!) à partir de \$100 sur les 12 mois glissants.

Ce document est un \WorkInProgressFR; par défaut les changements ne seront pas annoncés sinon via l'\href{https://github.com/erwannr/cerclef}{historique du dépôt}.
