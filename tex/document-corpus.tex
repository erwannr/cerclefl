% -----------------------------------------------------------------
% Copyright (C) 2025 by Le cercle floridien (Erwann Rogard)
% Source repository: https://github.com/erwannr/cerclefl
% 
% LaTeX code:
% Released under the LaTeX Project Public License v1.3c or later
% See http://www.latex-project.org/lppl.txt
% 
% Written content (text):
% CC BY-NC-SA 4.0
% -----------------------------------------------------------------

% Goes in the preamble:
% % -----------------------------------------------------------------
% Copyright (C) 2025 by La bergerie de Floride (Erwann Rogard)
% Source repository: https://github.com/erwannr/bergeriefl
% 
% LaTeX code:
% Released under the LaTeX Project Public License v1.3c or later
% See http://www.latex-project.org/lppl.txt
% 
% Written content (text):
% CC BY-NC-SA 4.0
% -----------------------------------------------------------------

\DTLnewdb{corpus-author}

\DTLnewrow{corpus-author}
\DTLnewdbentry{corpus-author}{id}{girard}
\DTLnewdbentry{corpus-author}{name}{René~Girard}
\DTLnewdbentry{corpus-author}{lifespan}{1923-2015}

\DTLnewrow{corpus-author}
\DTLnewdbentry{corpus-author}{id}{jodelle}
\DTLnewdbentry{corpus-author}{name}{Étienne~Jodelle}
\DTLnewdbentry{corpus-author}{lifespan}{1532-1573}

\DTLnewrow{corpus-author}
\DTLnewdbentry{corpus-author}{id}{montaigne}
\DTLnewdbentry{corpus-author}{name}{Michel~de~Montaigne}
\DTLnewdbentry{corpus-author}{lifespan}{1533-1592}

\DTLnewrow{corpus-author}
\DTLnewdbentry{corpus-author}{id}{stendhal}
\DTLnewdbentry{corpus-author}{name}{Stendhal}
\DTLnewdbentry{corpus-author}{lifespan}{1783-1842}

\DTLnewrow{corpus-author}
\DTLnewdbentry{corpus-author}{id}{voltaire}
\DTLnewdbentry{corpus-author}{name}{Voltaire}
\DTLnewdbentry{corpus-author}{lifespan}{1694-1778}

\DTLnewrow{corpus-author}
\DTLnewdbentry{corpus-author}{id}{zola}
\DTLnewdbentry{corpus-author}{name}{Émile Zola}
\DTLnewdbentry{corpus-author}{lifespan}{1840-1902}

% % -----------------------------------------------------------------
% Copyright (C) 2025 by La bergerie de Floride (Erwann Rogard)
% Source repository: https://github.com/erwannr/bergeriefl
% 
% LaTeX code:
% Released under the LaTeX Project Public License v1.3c or later
% See http://www.latex-project.org/lppl.txt
% 
% Written content (text):
% CC BY-NC-SA 4.0
% -----------------------------------------------------------------

\DTLnewdb{corpus-publication}

\DTLnewrow{corpus-publication}
\DTLnewdbentry{corpus-publication}{authorid}{jodelle}
\DTLnewdbentry{corpus-publication}{id}{amours}
\DTLnewdbentry{corpus-publication}{source}{\cite{jodelle2003amours}}
\DTLnewdbentry{corpus-publication}{read-id}{close}
\DTLnewdbentry{corpus-publication}{name}{Amours}
\DTLnewdbentry{corpus-publication}{keyword-id-clist}{renaiss,poetry,sonnet,pleiade}
\DTLnewdbentry{corpus-publication}{summary}{\WorkInProgressFR}

\DTLnewrow{corpus-publication}
\DTLnewdbentry{corpus-publication}{authorid}{stendhal}
\DTLnewdbentry{corpus-publication}{id}{lerouge}
\DTLnewdbentry{corpus-publication}{source}{\cite{matsumoto1997}\cite{ternois1937}}
\DTLnewdbentry{corpus-publication}{read-id}{open}
\DTLnewdbentry{corpus-publication}{name}{Le~rouge~et~le~noir}
\DTLnewdbentry{corpus-publication}{keyword-id-clist}{19c,mimetic-d,realism,restaur}
\DTLnewdbentry{corpus-publication}{summary}{\WorkInProgressFR}

\DTLnewrow{corpus-publication}
\DTLnewdbentry{corpus-publication}{authorid}{voltaire}
\DTLnewdbentry{corpus-publication}{id}{candide}
\DTLnewdbentry{corpus-publication}{source}{\cite{darcos2005}}
\DTLnewdbentry{corpus-publication}{read-id}{closed}
\DTLnewdbentry{corpus-publication}{name}{Candide}
\DTLnewdbentry{corpus-publication}{keyword-id-clist}{enlight,tale,philo}
\DTLnewdbentry{corpus-publication}{summary}{Une phrase clé de Candide, c'est la devise de Pangloss, le précepteur au château de Thunder-Ten-Tronckh : \enquote{Tout est pour le mieux dans le meilleur des mondes}. Tenant pour Candide de bagage philosophique, cet énoncé est mis à l'épreuve des faits au cours de périgrinations---catastrophes naturelles, autodafés, etc., prétexte à dénoncer les guerres de religion, superstitutions, etc. %\par{} % Bug?
  Pangloss incarnant, de manière parodique, un métaphysicien célèbre (un autre rationaliste que Descartes), cela traduit aussi, de la part de Voltaire, une saturation pour ce qu'il représente (que j'ai moi-même ressentie en essayant de m'y plonger\cite{parmentier2014}). Cependant, Voltaire et ses pairs ne jettent pas le bébé avec l'eau du bain ; autrement, Descartes ne serait pas considéré comme un père de la philosophie moderne\cite{losada2014}.}

% Removing it because it's a commentary on a classic not a classic itself
% \DTLnewrow{corpus-publication}
% \DTLnewdbentry{corpus-publication}{authorid}{girard}
% \DTLnewdbentry{corpus-publication}{id}{mensonge}
% \DTLnewdbentry{corpus-publication}{source}{\cite{girard1961}}
% \DTLnewdbentry{corpus-publication}{read-id}{pending}
% \DTLnewdbentry{corpus-publication}{name}{Mensonge~romantique~et~vérité~romanesque}
% \DTLnewdbentry{corpus-publication}{keyword-id-clist}{modern,mimetic-d,litercrit}
% \DTLnewdbentry{corpus-publication}{summary}{\WorkInProgressFR}

\DTLnewrow{corpus-publication}
\DTLnewdbentry{corpus-publication}{authorid}{montaigne}
\DTLnewdbentry{corpus-publication}{id}{essais}
\DTLnewdbentry{corpus-publication}{source}{\cite{pernon2009}\cite{lenuki2014-10-19}}
\DTLnewdbentry{corpus-publication}{read-id}{pause}
\DTLnewdbentry{corpus-publication}{name}{Les~essais~--~Livre~I}
\DTLnewdbentry{corpus-publication}{keyword-id-clist}{renaiss,essai,philo,classicism}
\DTLnewdbentry{corpus-publication}{summary}{\WorkInProgressFR}

% % -----------------------------------------------------------------
% Copyright (C) 2025 by Le cercle floridien (Erwann Rogard)
% Source repository: https://github.com/erwannr/cerclefl
% 
% LaTeX code:
% Released under the LaTeX Project Public License v1.3c or later
% See http://www.latex-project.org/lppl.txt
% 
% Written content (text):
% CC BY-NC-SA 4.0
% -----------------------------------------------------------------

\DTLnewdb{corpus-keyword}

\DTLnewrow{corpus-keyword}
\DTLnewdbentry{corpus-keyword}{id}{litercrit}
\DTLnewdbentry{corpus-keyword}{name}{Critique~littéraire}

\DTLnewrow{corpus-keyword}
\DTLnewdbentry{corpus-keyword}{id}{essay}
\DTLnewdbentry{corpus-keyword}{name}{essai}

\DTLnewrow{corpus-keyword}
\DTLnewdbentry{corpus-keyword}{id}{mimetic-d}
\DTLnewdbentry{corpus-keyword}{name}{désir~mimétique\cite{girard1961}}

\DTLnewrow{corpus-keyword}
\DTLnewdbentry{corpus-keyword}{id}{philo}
\DTLnewdbentry{corpus-keyword}{name}{philosophie}

\DTLnewrow{corpus-keyword}
\DTLnewdbentry{corpus-keyword}{id}{poetry}
\DTLnewdbentry{corpus-keyword}{name}{poésie}

\DTLnewrow{corpus-keyword}
\DTLnewdbentry{corpus-keyword}{id}{pleiade}
\DTLnewdbentry{corpus-keyword}{name}{la~Pléiade}

\DTLnewrow{corpus-keyword}
\DTLnewdbentry{corpus-keyword}{id}{realism}
\DTLnewdbentry{corpus-keyword}{name}{Réalisme}

\DTLnewrow{corpus-keyword}
\DTLnewdbentry{corpus-keyword}{id}{restaur}
\DTLnewdbentry{corpus-keyword}{name}{la~Restauration}

\DTLnewrow{corpus-keyword}
\DTLnewdbentry{corpus-keyword}{id}{sonnet}
\DTLnewdbentry{corpus-keyword}{name}{sonnet}

\DTLnewrow{corpus-keyword}
\DTLnewdbentry{corpus-keyword}{id}{tale}
\DTLnewdbentry{corpus-keyword}{name}{conte}

\DTLnewrow{corpus-keyword}
\DTLnewdbentry{corpus-keyword}{id}{19c}
\DTLnewdbentry{corpus-keyword}{name}{\OrdinalAbbrevFR{XIX}}

\DTLnewrow{corpus-keyword}
\DTLnewdbentry{corpus-keyword}{id}{enlight}
\DTLnewdbentry{corpus-keyword}{name}{Siècle~des~Lumières}

\DTLnewrow{corpus-keyword}
\DTLnewdbentry{corpus-keyword}{id}{modern}
\DTLnewdbentry{corpus-keyword}{name}{Époque moderne}

\DTLnewrow{corpus-keyword}
\DTLnewdbentry{corpus-keyword}{id}{renaiss}
\DTLnewdbentry{corpus-keyword}{name}{Rennaissance}

\DTLnewrow{corpus-keyword}
\DTLnewdbentry{corpus-keyword}{id}{classicism}
\DTLnewdbentry{corpus-keyword}{name}{Classicisme}

% \input{../tex/corpus-lerouge.tex}

\part{Cahier des savoirs}
\label{part-corpus}

\chapter{Œuvre}
\label{corpus-publ}
\CorpusPublicationTypeset{amours}
\CorpusPublicationTypeset{candide}
\CorpusPublicationTypeset{essais}
\CorpusPublicationTypeset{lerouge}  
\CorpusPublicationTypeset{mensonge}

\chapter{Divers}
\label{corpus-misc}

\section{Rationalisme et foi}
\label{corpus-misc-foi}
Au détour de Candide, nous avons évoqué René Descartes et Blaise Pascal. L'objectif : à partir du \enquote{ je pense donc je suis } et du roseau pensant, déplier le débat entre rationalisme et foi. Pour en saisir la portée contemporaine, le cinéma de la Nouvelle Vague nous offre une piste\cite{wiki:Ma_nuit_chez_Maud}.
