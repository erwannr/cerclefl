% -----------------------------------------------------------------
% Copyright (C) 2025 by Le cercle floridien (Erwann Rogard)
% Source repository: https://github.com/erwannr/cerclefl
% 
% LaTeX code:
% Released under the LaTeX Project Public License v1.3c or later
% See http://www.latex-project.org/lppl.txt
% 
% Written content (text):
% CC BY-NC-SA 4.0
% -----------------------------------------------------------------

\DTLnewdb{corpus-keyword}

\DTLnewrow{corpus-keyword}
\DTLnewdbentry{corpus-keyword}{id}{litercrit}
\DTLnewdbentry{corpus-keyword}{name}{Critique~littéraire}

\DTLnewrow{corpus-keyword}
\DTLnewdbentry{corpus-keyword}{id}{essay}
\DTLnewdbentry{corpus-keyword}{name}{essai}

\DTLnewrow{corpus-keyword}
\DTLnewdbentry{corpus-keyword}{id}{mimetic-d}
\DTLnewdbentry{corpus-keyword}{name}{désir~mimétique\cite{girard1961}}

\DTLnewrow{corpus-keyword}
\DTLnewdbentry{corpus-keyword}{id}{philo}
\DTLnewdbentry{corpus-keyword}{name}{philosophie}

\DTLnewrow{corpus-keyword}
\DTLnewdbentry{corpus-keyword}{id}{poetry}
\DTLnewdbentry{corpus-keyword}{name}{poésie}

\DTLnewrow{corpus-keyword}
\DTLnewdbentry{corpus-keyword}{id}{pleiade}
\DTLnewdbentry{corpus-keyword}{name}{la~Pléiade}

\DTLnewrow{corpus-keyword}
\DTLnewdbentry{corpus-keyword}{id}{realism}
\DTLnewdbentry{corpus-keyword}{name}{Réalisme}

\DTLnewrow{corpus-keyword}
\DTLnewdbentry{corpus-keyword}{id}{restaur}
\DTLnewdbentry{corpus-keyword}{name}{la~Restauration}

\DTLnewrow{corpus-keyword}
\DTLnewdbentry{corpus-keyword}{id}{sonnet}
\DTLnewdbentry{corpus-keyword}{name}{sonnet}

\DTLnewrow{corpus-keyword}
\DTLnewdbentry{corpus-keyword}{id}{tale}
\DTLnewdbentry{corpus-keyword}{name}{conte}

\DTLnewrow{corpus-keyword}
\DTLnewdbentry{corpus-keyword}{id}{19c}
\DTLnewdbentry{corpus-keyword}{name}{\OrdinalAbbrevFR{XIX}}

\DTLnewrow{corpus-keyword}
\DTLnewdbentry{corpus-keyword}{id}{enlight}
\DTLnewdbentry{corpus-keyword}{name}{Siècle~des~Lumières}

\DTLnewrow{corpus-keyword}
\DTLnewdbentry{corpus-keyword}{id}{modern}
\DTLnewdbentry{corpus-keyword}{name}{Époque moderne}

\DTLnewrow{corpus-keyword}
\DTLnewdbentry{corpus-keyword}{id}{renaiss}
\DTLnewdbentry{corpus-keyword}{name}{Rennaissance}

\DTLnewrow{corpus-keyword}
\DTLnewdbentry{corpus-keyword}{id}{classicism}
\DTLnewdbentry{corpus-keyword}{name}{Classicisme}
