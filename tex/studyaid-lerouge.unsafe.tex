% -----------------------------------------------------------------
% Copyright (C) 2025 by La bergerie de Floride (Erwann Rogard)
% Source repository: https://github.com/erwannr/bergeriefl
% 
% LaTeX code:
% Released under the LaTeX Project Public License v1.3c or later
% See http://www.latex-project.org/lppl.txt
% 
% Written content (text):
% CC BY-NC-SA 4.0
% -----------------------------------------------------------------

\CounterSetup{ new = lerouge-unsafe, new = lerouge-safe }

\DTLnewdb{studyaid-lerouge}

\CounterIncrement{lerouge-unsafe}
% <uncensor>
\CounterIncrement{lerouge-safe}
% </uncensor>
\DTLnewrow{studyaid-lerouge}
\DTLnewdbentry{studyaid-lerouge}{id}{tous-ces-marchands}
\DTLnewdbentry{studyaid-lerouge}{question}{Comment le caractère de M. de Rênal se révèle-t-il dans ce chapitre ? Quels sont les divers mobiles qui le poussent à donner un précepteur à ses enfants ? Dans quel ordre présente-t-il ses arguments à sa femme?\cite{ternois1937}}
\DTLnewdbentry{studyaid-lerouge}{exhibit}{\CorpusQuoteTypeset{lerouge-tous-ces-marchands}}
\DTLnewdbentry{studyaid-lerouge}{answer}{%
  % <uncensor>
  M. de Rênal recrute Sorel comme précepteur pour soutenir leur rang. En tant que réactionnaire, il rationalise ouvertement d'accepter chez lui le neveu d'un libéral (le vieux chirurgien-major), c'est à dire un ennemi idéologique. C'est, au figuré, un \emph{pharisien}\cite{larousse-pharisien}\cite{pascal-fragment-469}.
  % </uncensor>
}

\CounterIncrement{lerouge-unsafe}
% <censor>
\CounterIncrement{lerouge-safe}
% </censor>
\DTLnewrow{studyaid-lerouge}
\DTLnewdbentry{studyaid-lerouge}{id}{ame-naive}
\DTLnewdbentry{studyaid-lerouge}{question}{%
  % <uncensor>
  \CensoredElement
  % </uncensor>
  % <censor>
  Vous comparerez \Mme{de Rênal} et \Mme{de Chasteller} dans la première partie de Lucien Lesasen.  
  Vous comparerez \Mme{de Rênal} et \Mme{Bovary}.
  \cite{ternois1937}
  % </censor>
}
\DTLnewdbentry{studyaid-lerouge}{exhibit}{\CorpusQuoteTypeset{lerouge-ame-naive}}
\DTLnewdbentry{studyaid-lerouge}{answer}{%
  % <uncensor>
  \CensoredElement
  % </uncensor>
  % <censor>
  Toutes les deux élevées au couvent, \Mme{de Rênal} est naïve; \Mme{Bovary} est romantique\endnote{Parallèlement, M. Bovary est plat, tandis que M. de Rênal est vaniteux.}.
  % </censor>
}

\CounterIncrement{lerouge-unsafe}
% <censor>
\CounterIncrement{lerouge-safe}
% </censor>
\DTLnewrow{studyaid-lerouge}
\DTLnewdbentry{studyaid-lerouge}{id}{julien-lisait}
\DTLnewdbentry{studyaid-lerouge}{question}{%
  % <uncensor>
  \CensoredElement
  % </uncensor>
  % <censor>
  Pourquoi Stendhal, habituellement si sobre de détails descriptifs, donne-t-il ici tant d'importance au décor\cite{ternois1937}?
  % </censor>
}
\DTLnewdbentry{studyaid-lerouge}{exhibit}{\CorpusQuoteTypeset{lerouge-julien-lisait}}
\DTLnewdbentry{studyaid-lerouge}{answer}{%
  % <uncensor>
  \CensoredElement
  % </uncensor>
  % <censor>
  Monde rural = de taiseux; d'où la description du décor.
  % </censor>
}

\CounterIncrement{lerouge-unsafe}
% <censor>
\CounterIncrement{lerouge-safe}
% </censor>
\DTLnewrow{studyaid-lerouge}
\DTLnewdbentry{studyaid-lerouge}{id}{eh-bien-paresseux}
\DTLnewdbentry{studyaid-lerouge}{question}{%
  % <uncensor>
  \CensoredElement
  % </uncensor>
  % <censor>
  Le caractère du père Sorel. Pourquoi hait-il son fils ?  
  Vous lirez dans la Vie de Henri Brulard ce que Stendhal dit de son père et vous vous demanderez dans quelle mesure il s'est souvenu de lui pour représenter le père Sorel.  
  N'y a-t-il pas quelque ressemblance entre le père Sorel et le père Grandet ?
  % </censor>
}
\DTLnewdbentry{studyaid-lerouge}{exhibit}{\CorpusQuoteTypeset{lerouge-eh-bien-paresseux}}
\DTLnewdbentry{studyaid-lerouge}{answer}{%
  % <uncensor>
  \CensoredElement
  % </uncensor>
  % <censor>
  Intellectuel parmi les ruraux---ressentiment.
  % </censor>
}

\CounterIncrement{lerouge-unsafe}
% <censor>
\CounterIncrement{lerouge-safe}
% </censor>
\DTLnewrow{studyaid-lerouge}
\DTLnewdbentry{studyaid-lerouge}{id}{legue-sa-croix}
\DTLnewdbentry{studyaid-lerouge}{question}{%
  % <uncensor>
  \CensoredElement
  % </uncensor>
  % <censor>
  Pourquoi le vieux chirurgien-major s'intéressait-il à Julien ? 
  % </censor>
}
\DTLnewdbentry{studyaid-lerouge}{exhibit}{\CorpusQuoteTypeset{lerouge-legue-sa-croix}}
\DTLnewdbentry{studyaid-lerouge}{answer}{%
  % <uncensor>
  \CensoredElement
  % </uncensor>
  % <censor>
  Le vieux chirurgien-major s'intéresse à Julien parce qu'il voit en lui un disciple.
  % </censor>
}

\CounterIncrement{lerouge-unsafe}
% <censor>
\CounterIncrement{lerouge-safe}
% </censor>
\DTLnewrow{studyaid-lerouge}
\DTLnewdbentry{studyaid-lerouge}{id}{ne-vois-que-dieu}
\DTLnewdbentry{studyaid-lerouge}{question}{%
  % <uncensor>
  \CensoredElement
  % </uncensor>
  % <censor>
  Pourquoi le père Sorel suppose-t-il à ce que son fils connaisse \Mme{de Rênal} ? 
  % </censor>
}
\DTLnewdbentry{studyaid-lerouge}{exhibit}{\CorpusQuoteTypeset{lerouge-ne-vois-que-dieu}}
\DTLnewdbentry{studyaid-lerouge}{answer}{%
  % <uncensor>
  \CensoredElement
  % </uncensor>
  % <censor>
  Humiliation habituelle.
  % </censor>
}

\CounterIncrement{lerouge-unsafe}
% <censor>
\CounterIncrement{lerouge-safe}
% </censor>
\DTLnewrow{studyaid-lerouge}
\DTLnewdbentry{studyaid-lerouge}{id}{pas-etre-domestique}
\DTLnewdbentry{studyaid-lerouge}{question}{%
  % <uncensor>
  \CensoredElement
  % </uncensor>
  % <censor>
  Que vous révèlent cette réplique de Julien?
  % </censor>
}
\DTLnewdbentry{studyaid-lerouge}{exhibit}{\CorpusQuoteTypeset{lerouge-pas-etre-domestique}}
\DTLnewdbentry{studyaid-lerouge}{answer}{%
  % <uncensor>
  \CensoredElement
  % </uncensor>
  % <censor>
  Conscience sociale du milieu et snobisme de Julien.
  % </censor>
}

\CounterIncrement{lerouge-unsafe}
% <censor>
\CounterIncrement{lerouge-safe}
% </censor>
\DTLnewrow{studyaid-lerouge}
\DTLnewdbentry{studyaid-lerouge}{id}{caractere}
\DTLnewdbentry{studyaid-lerouge}{question}{%
  % <uncensor>
  \CensoredElement
  % </uncensor>    
  % <censor>
  Vous montrerez comment dans ce chapitre se dessine le caractère de Julien.  
  Vous chercherez dans la suite du roman quels ont été les effets de son imagination.
  % </censor>
}
\DTLnewdbentry{studyaid-lerouge}{exhibit}{}
\DTLnewdbentry{studyaid-lerouge}{answer}{%
  % <uncensor>
  \CensoredElement
  % </uncensor>    
  % <censor>
  Ambitieux, hypocrite, imaginatif, orgueilleux.
  % </censor>
}

\CounterIncrement{lerouge-unsafe}
% <censor>
\CounterIncrement{lerouge-safe}
% </censor>
\DTLnewrow{studyaid-lerouge}
\DTLnewdbentry{studyaid-lerouge}{id}{influence}
\DTLnewdbentry{studyaid-lerouge}{question}{%
  % <uncensor>
  \CensoredElement
  % </uncensor>    
  % <censor>
  Vous vous demanderez, en lisant les chapitres suivants, quelle influence ont eu sur Julien les Confessions et le Mémorial de Sainte-Hélène.
  % </censor>
}
\DTLnewdbentry{studyaid-lerouge}{exhibit}{}
\DTLnewdbentry{studyaid-lerouge}{answer}{%
  % <uncensor>
  \CensoredElement
  % </uncensor>    
  % <censor>
  Des Confessions il tient le snobisme (?) et du mémorial de Saint-Hélène l'ambition.
  % </censor>
}

\CounterIncrement{lerouge-unsafe}
% <censor>
\CounterIncrement{lerouge-safe}
% </censor>
\DTLnewrow{studyaid-lerouge}
\DTLnewdbentry{studyaid-lerouge}{id}{gagner-le-vieux-cure}
\DTLnewdbentry{studyaid-lerouge}{question}{%
  % <uncensor>
  \CensoredElement
  % </uncensor>    
  % <censor>
  Pourquoi Julien a-t-il appris par cœur le Pape de Joseph de Maistre ?
  % </censor>
}
\DTLnewdbentry{studyaid-lerouge}{exhibit}{\CorpusQuoteTypeset{lerouge-gagner-le-vieux-cure}}
\DTLnewdbentry{studyaid-lerouge}{answer}{%
  % <uncensor>
  \CensoredElement
  % </uncensor>    
  % <censor>
  Julien a appris par cœur le Pape de Joseph de Maistre pour gagner le curé Chélan en qui il voit le moyen de son ambition.
  % </censor>
}
