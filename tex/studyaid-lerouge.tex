% -----------------------------------------------------------------
% Copyright (C) 2025 by Le cercle floridien (Erwann Rogard)
% Source repository: https://github.com/erwannr/cerclefl
% 
% LaTeX code:
% Released under the LaTeX Project Public License v1.3c or later
% See http://www.latex-project.org/lppl.txt
% 
% Written content (text):
% CC BY-NC-SA 4.0
% -----------------------------------------------------------------

\CounterSetup{ new = lerouge-unsafe, new = lerouge-safe }

\DTLnewdb{studyaid-lerouge}

\CounterIncrement{lerouge-unsafe}
% <uncensor>
\CounterIncrement{lerouge-safe}
% </uncensor>
\DTLnewrow{studyaid-lerouge}
\DTLnewdbentry{studyaid-lerouge}{id}{tous-ces-marchands}
\DTLnewdbentry{studyaid-lerouge}{question}{Comment le caractère de M. de Rênal se révèle-t-il dans ce chapitre ? Quels sont les divers mobiles qui le poussent à donner un précepteur à ses enfants ? Dans quel ordre présente-t-il ses arguments à sa femme?\cite{ternois1937}}
\DTLnewdbentry{studyaid-lerouge}{exhibit}{\CorpusQuoteTypeset{lerouge-tous-ces-marchands}}
\DTLnewdbentry{studyaid-lerouge}{answer}{%
  % <uncensor>
  M. de Rênal recrute Sorel comme précepteur pour soutenir leur rang. En tant que réactionnaire, il rationalise ouvertement d'accepter chez lui le neveu d'un libéral (le vieux chirurgien-major), c'est à dire un ennemi idéologique. C'est, au figuré, un \emph{pharisien}\cite{larousse-pharisien}\cite{pascal-fragment-469}.
  % </uncensor>
}

\CounterIncrement{lerouge-unsafe}
% CENSORED
\DTLnewrow{studyaid-lerouge}
\DTLnewdbentry{studyaid-lerouge}{id}{ame-naive}
\DTLnewdbentry{studyaid-lerouge}{question}{%
  % <uncensor>
  \CensoredElementFR
  % </uncensor>
  % CENSORED
}
\DTLnewdbentry{studyaid-lerouge}{exhibit}{\CorpusQuoteTypeset{lerouge-ame-naive}}
\DTLnewdbentry{studyaid-lerouge}{answer}{%
  % <uncensor>
  \CensoredElementFR
  % </uncensor>
  % CENSORED
}

\CounterIncrement{lerouge-unsafe}
% CENSORED
\DTLnewrow{studyaid-lerouge}
\DTLnewdbentry{studyaid-lerouge}{id}{julien-lisait}
\DTLnewdbentry{studyaid-lerouge}{question}{%
  % <uncensor>
  \CensoredElementFR
  % </uncensor>
  % CENSORED
}
\DTLnewdbentry{studyaid-lerouge}{exhibit}{\CorpusQuoteTypeset{lerouge-julien-lisait}}
\DTLnewdbentry{studyaid-lerouge}{answer}{%
  % <uncensor>
  \CensoredElementFR
  % </uncensor>
  % CENSORED
}

\CounterIncrement{lerouge-unsafe}
% CENSORED
\DTLnewrow{studyaid-lerouge}
\DTLnewdbentry{studyaid-lerouge}{id}{eh-bien-paresseux}
\DTLnewdbentry{studyaid-lerouge}{question}{%
  % <uncensor>
  \CensoredElementFR
  % </uncensor>
  % CENSORED
}
\DTLnewdbentry{studyaid-lerouge}{exhibit}{\CorpusQuoteTypeset{lerouge-eh-bien-paresseux}}
\DTLnewdbentry{studyaid-lerouge}{answer}{%
  % <uncensor>
  \CensoredElementFR
  % </uncensor>
  % CENSORED
}

\CounterIncrement{lerouge-unsafe}
% CENSORED
\DTLnewrow{studyaid-lerouge}
\DTLnewdbentry{studyaid-lerouge}{id}{legue-sa-croix}
\DTLnewdbentry{studyaid-lerouge}{question}{%
  % <uncensor>
  \CensoredElementFR
  % </uncensor>
  % CENSORED
}
\DTLnewdbentry{studyaid-lerouge}{exhibit}{\CorpusQuoteTypeset{lerouge-legue-sa-croix}}
\DTLnewdbentry{studyaid-lerouge}{answer}{%
  % <uncensor>
  \CensoredElementFR
  % </uncensor>
  % CENSORED
}

\CounterIncrement{lerouge-unsafe}
% CENSORED
\DTLnewrow{studyaid-lerouge}
\DTLnewdbentry{studyaid-lerouge}{id}{ne-vois-que-dieu}
\DTLnewdbentry{studyaid-lerouge}{question}{%
  % <uncensor>
  \CensoredElementFR
  % </uncensor>
  % CENSORED
}
\DTLnewdbentry{studyaid-lerouge}{exhibit}{\CorpusQuoteTypeset{lerouge-ne-vois-que-dieu}}
\DTLnewdbentry{studyaid-lerouge}{answer}{%
  % <uncensor>
  \CensoredElementFR
  % </uncensor>
  % CENSORED
}

\CounterIncrement{lerouge-unsafe}
% CENSORED
\DTLnewrow{studyaid-lerouge}
\DTLnewdbentry{studyaid-lerouge}{id}{pas-etre-domestique}
\DTLnewdbentry{studyaid-lerouge}{question}{%
  % <uncensor>
  \CensoredElementFR
  % </uncensor>
  % CENSORED
}
\DTLnewdbentry{studyaid-lerouge}{exhibit}{\CorpusQuoteTypeset{lerouge-pas-etre-domestique}}
\DTLnewdbentry{studyaid-lerouge}{answer}{%
  % <uncensor>
  \CensoredElementFR
  % </uncensor>
  % CENSORED
}

\CounterIncrement{lerouge-unsafe}
% CENSORED
\DTLnewrow{studyaid-lerouge}
\DTLnewdbentry{studyaid-lerouge}{id}{caractere}
\DTLnewdbentry{studyaid-lerouge}{question}{%
  % <uncensor>
  \CensoredElementFR
  % </uncensor>    
  % CENSORED
}
\DTLnewdbentry{studyaid-lerouge}{exhibit}{\IntentionallyBlankFR}
\DTLnewdbentry{studyaid-lerouge}{answer}{%
  % <uncensor>
  \CensoredElementFR
  % </uncensor>    
  % CENSORED
}

\CounterIncrement{lerouge-unsafe}
% CENSORED
\DTLnewrow{studyaid-lerouge}
\DTLnewdbentry{studyaid-lerouge}{id}{influence}
\DTLnewdbentry{studyaid-lerouge}{question}{%
  % <uncensor>
  \CensoredElementFR
  % </uncensor>    
  % CENSORED
}
\DTLnewdbentry{studyaid-lerouge}{exhibit}{\IntentionallyBlankFR}
\DTLnewdbentry{studyaid-lerouge}{answer}{%
  % <uncensor>
  \CensoredElementFR
  % </uncensor>    
  % CENSORED
}

\CounterIncrement{lerouge-unsafe}
% CENSORED
\DTLnewrow{studyaid-lerouge}
\DTLnewdbentry{studyaid-lerouge}{id}{gagner-le-vieux-cure}
\DTLnewdbentry{studyaid-lerouge}{question}{%
  % <uncensor>
  \CensoredElementFR
  % </uncensor>    
  % CENSORED
}
\DTLnewdbentry{studyaid-lerouge}{exhibit}{\CorpusQuoteTypeset{lerouge-gagner-le-vieux-cure}}
\DTLnewdbentry{studyaid-lerouge}{answer}{%
  % <uncensor>
  \CensoredElementFR
  % </uncensor>    
  % CENSORED
}
